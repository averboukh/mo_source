\documentclass[a4paper,12pt]{report}
\usepackage[russian]{ayv_lecturestyle}
%\usepackage[displaymath,mathlines]{lineno}
%\linenumbers

\begin{document}
\title{Методы оптимизации I семестр 4-й курс, МТ УрФУ. Конспект лекций}
\author{Лектор: Ю.В. Авербух}
\maketitle

\newcommand{\rd}{\mathbb{R}^d}
\newcommand{\rdc}{\mathbb{R}^{d*}}

\chapter*{Вводные замечания}

Методы оптимизации -- курс предмет, которого невероятно широк. Собственно говоря, методы оптимизации занимаются всем тем, где надо найти экстремум (т.е. минимум или максимум)  некоторой функции при тех или иных ограничениях. Зачастую, чем более широкие ограничения наложены, тем содержательнее получаются условия на экстремальную точку. Предполагается, что мы изучим условия экстремума функции с ограничениями и задачу поиска экстремума функционала в классе непрерывно дифференцируемых функций. Последняя задача традиционно выделяется в теорию вариационного исчисления и играет важную роль в механике и физике.

\section*{Список обозначений}
\noindent $\triangleq$ -- равно по определению.

\noindent $\mathbb{R}$ -- вещественная прямая.

\noindent $\mathbb{R}^d$ -- $d$-мерное евклидово пространство. Будем считать, что элементы $\mathbb{R}^d$ -- вектор-столбцы.

\noindent $\mathbb{R}^{d*}$ -- пространство, сопряженное к $\mathbb{R}^d$, его элементы вектор-строки.

\noindent $O_\varepsilon(x)$ -- окрестность $x$.

\noindent $\mathrm{co}(Y)$ -- выпуклая оболочка множества $Y$. 

\section*{Функции и пространства}
Напомним, что нормированным пространством называют пару $(X,\|\cdot\|)$ такую, что $X$ -- линейное пространство, а $\|\cdot\|$ -- норма на $X$ т.е. функция из $X$ в $[0,+\infty)$, удовлетворяющая условиям
\begin{enumerate}
	\item $\|x\|=0$ тогда и только тогда, когда $x=0$;
	\item $\|\alpha x\|=|\alpha| \|x\|$;
	\item (неравенство треугольника) $\|x+y\|\leq \|x\|+\|y\|$.
\end{enumerate} Норма задает сходимость. А именно, говорят, что последовательность $\{x_n\}_{n=1}^\infty$ сходится к $x$, если $\|x_n-x\|$ стремится к 0. Далее мы будем предполагать, что $X$ -- банахово, а именно мы считаем, что $X$ полно (т.е. любая фундаментальная последовательность имеет предел) и сепарабельно (т.е. существует счетное всюду плотное подмножество $X$). Если $x\in X$, $\varepsilon>0$, то через обозначим $\varepsilon$-окрестность $x$:
$$O_\varepsilon(x)\triangleq \{y\in X:\|y-x\|<\varepsilon\}. $$ 

Если $Y$ -- также банахово пространство, то множество линейных непрерывных операторов из $X$ в $Y$ обозначим через $\mathcal{L}(X,Y)$. Если $A\in\mathcal{L}(X,Y)$, то введя т.н. сильную норму
$$\|A\|\triangleq \sup\left\{\frac{\|Ax\|}{\|x\|}:x\in X\right\}=\sup\{\|Ax\|:x\in X,\ \ \|x\|\leq 1\}, $$ мы получаем, что $\mathcal{L}(X,Y)$ также является банаховым.


С пространством $X$ связывают пространство его линейных функционалов из $X$ в $\mathbb{R}$ т.е. пространство $\mathcal{L}(X,\rd)$. Это пространство называют сопряженным к $X$ и обозначают $X^*$. Пространство $X^*$ само является линейным. Его можно нормировать несколькими способами. Если $p\in X^*$, $x\in X$, то значение линейной функции $p(x)$ также обозначают $\langle p,x\rangle$.

Для нас основными примерами банаховых пространств служат: конечномерное евклидово пространство и пространства непрерывных функций. Рассмотрим их поподробнее. 

Конечномерное евклидово пространство для нас -- это пространство $\rd$, состоящее из $d$-мерных векторов $x$. В дальнейшем мы примем соглашение, что $x$ -- это вектор столбец:
$$x=\left(\begin{array}{c}
x_1 \\ x_2 \\ \cdots \\ x_d
\end{array}\right). $$
Норма в $\rd$ определяется$$x=\left(\begin{array}{c}
x_1 \\ x_2 \\ \cdots \\ x_d
\end{array}\right). $$ стандартным образом $$\|x\|=\sqrt{x^T x}=\sqrt{\sum_{i=1}^d x_i}.$$ Здесь и далее $x^T$ -- обозначает транспонирование. Из курса линейной алгебры известно, что всякий линейный функционал $\phi$ из $\rd$ в $\mathbb{R}$ может быть представлен вектор-строкой. В дальнейшем мы просто считаем, что элементы $\rdc$ суть вектор строки и будем писать $px$ вместо $\langle p,x\rangle$. 

Особо важную роль будут играть пространства непрерывных и непрерывно дифференцируемых функций одномерного аргумента (времени). Несколько неточно мы будем обозначать функции времени указывая наличие аргумента т.е. в дальнейшем всегда $x(\cdot)$ -- некоторая функция времени. Также $x(t)$ всегда обозначает значение $x(\cdot)$ в точке $t$. 

Пусть $a,b\in \mathbb{R}$, $a<b$. Можно считать, что это какие-то моменты времени. Обозначим, через $C([a,b];\rd)$ -- множество непрерывных функций из $[a,b]$ в $\rd$. В некоторых случаях нам будут нужны координатное представление т.е. мы представляем $x(\cdot)$ в качестве набора $d$ непрерывных функций
$$x(\cdot)=\left(\begin{array}{c}
x_1(\cdot) \\ x_2(\cdot) \\ \cdots \\ x_d(\cdot)
\end{array}\right). $$ На $C([a,b];\rd)$ вводится $\sup$-норма т.е. $$\|x(\cdot)\|_0=\sup_{t\in [a,b]}\|x(t)\|.$$ Этот супремум достигается. Пространство $C([a,b];\rd)$ с нормой $\|\cdot\|_0$ является банаховым.

В курсе функционального анализа доказывается, что сопряженным к пространству $C([a,b];\rd)$ является пространство функций ограниченной вариации, определенных на $[a,b]$ со значениями в $\rd$. 

Если $x(\cdot)$ функция времени со значениями в $\rd$, то производную по времени будем обозначать точкой сверху т.е.
$$\dot{x}(t)=\frac{d}{dt}x(t). $$ Функция, равная в каждой точке производной $x(t)$ (если она корректно определена) обозначим через $\dot{x}(\cdot)$.

Далее, через $C^1([a,b];\rd)$ обозначают пространство непрерывно дифференцируемых функций из $[a,b]$ в $\rd$. Норму на $C^1([a,b];\rd)$ определим по формуле
$$\|x(\cdot)\|=\sup_{t\in [a,b]}\|x(t)\|+\sup_{t\in [a,b]}\|\dot{x}(t)\|=\|x(\cdot)\|_0+\|\dot{x}(\cdot)\|_0. $$ Пространство $C^1([a,b];\rd)$ с нормой $\|\cdot\|_1$ является банаховым. Отметим, что $C^1([a,b];\rd)\subset C([a,b];\rd)$. В то же время пространство $C^1([a,b];\rd)$, оснащенное нормой $\|\cdot\|_0$, не полно (проверьте!).

Напомним, что $k$-ю производную принято обозначать верхним индексом в скобках. Пространство $n$-раз непрерывно дифференцируемых функций времени обозначим через $C^k([a,b];\rd)$. Норму на $C^n([a,b];\rd)$ введем по правилу:
$$\|x(\cdot)\|_n\triangleq \sum_{k=1}^n \|x^{(k)}(\cdot)\|_0. $$

Наконец, пусть $F$ -- произвольная функция из $X$ в $Y$, где $X$ и $Y$ -- некоторые банаховы пространства. Будем говорить, что $F$ дифференцируема по Гато в $x\in X$, если существует непрерывные линейный оператор $A$ из $X$ в $Y$ такой, что для всякого $h\in X$,
$$\lim_{t\downarrow 0}\frac{F(x+t h)}{t}=Ah. $$ При этом $A$ называется производной по Гато. Если же мы можем представить $F(x+th)$ в виде
$$F(x+th)=F(x)+tAh+r(x,h), $$ где $\|r(x,h)\|/\|h\|\rightarrow 0$ при $\|h\|\rightarrow 0$, то говорят, что $F$ дифференцируема по Фреше. Заметим, что дифференцируемость по Фреше влечет дифференцируемость по Гато, при этом производные совпадают. Для конечномерных пространств оба понятия всегда совпадают. В то же время понятие дифференцируемости позволяет прийти к важному замечанию. Пусть $F:\rd\rightarrow\mathbb{R}$. Тогда производная есть элемент $\mathcal{L}(\rd,\mathbb{R})=\rdc$, т.е. производная  скалярной функции векторного аргумента есть вектор-строка.

\chapter{Необходимые условия минимума функции с ограничениями в виде неравенств}
\section{Безусловная оптимизация}

Необходимые условия безусловной оптимизации были предложены еще Пьером Ферма. Постановка задачи очень проста. Пусть $X$ -- банахово пространство, $\phi:X\rightarrow \mathbb{R}$. Надо найти необходимое условие того, что в $x_*$ достигается локальный экстремум.
\begin{theorem}[принцип Ферма]\label{th:Fermat}
 Пусть $x_*$ точка локального экстремума. Предположим, что $\phi$ дифференцируемо по Гато в $x_*$, $\phi'(x_*)$ обозначает производную по Гато. Тогда
 $$ \phi'(x)=0.$$
\end{theorem} 

Доказательство этой теоремы в конечномерном случае было в курсе математического анализа. Доказательство для случая банахова пространства предоставляется читателю в качестве несложного упражнения.
\begin{task}(0.1 балл) Докажите принципе Ферма.
\end{task}

\section{Условная оптимизация}
Первые задачи условной оптимизации с ограничениями в виде равенств были решены Лагранжем. Дальнейшее развитие связано с теоремой Куна-Такера, в которой рассматриваются ограничения в виде неравенств.

Для определенности мы рассмотрим задачу минимизации функции $\phi:X\rightarrow\mathbb{R}$ в множестве, задаваемом системой $m$ неравенств: $\psi_i(x)\leq 0$. Здесь $\psi_i:X\rightarrow \mathbb{R}$. В дальнейшем обозначим $\psi(x)\triangleq (\psi_1(x),\ldots,\psi_m(x))^T$. Таким образом, $\psi$ -- это функция из $X$ в $\mathbb{R}^m$. Задача минимизации формулируется в векторном виде
\begin{equation}\label{problem:kuhn_tacker_statement}
\text{минимизировать } \phi(x) 
\end{equation}
\begin{equation}\label{problem:kuhn_tacker_conditions}
\text{при условии }\psi(x)\leq 0. 
\end{equation} В последнем случае $0$ -- это вектор, составленный из нулей, а неравенство надо понимать покоординатно.
\begin{definition}\label{def:cond_minimum} Будем говорить, что $x_*$ локальный минимум функции $\phi$ при условиях, задаваемых системой неравенств $\psi(x)\leq 0$, если $x_*$ допустима и существует $\varepsilon>0$ такое, что для каждой $x\in O_\varepsilon(x_*)$ из условий $\psi(x)\leq 0$ следует, что $\phi(x)\geq \phi(x_*)$. 
\end{definition}

 Мы дадим лишь необходимое условие локального минимума в виде теоремы Куна-Такера, следуя \cite{Kuhn_Tacker}.

\begin{theorem}[Кун-Такер]\label{th:Kuhn_Tacker}
	Пусть $x_*$ локальный минимум в задаче (\ref{problem:kuhn_tacker_statement}), (\ref{problem:kuhn_tacker_conditions}). Предположим, что  функции $\phi$ и $\psi_i$ непрерывно дифференцируемы в $x_*$. Тогда существуют  $\lambda_0$, $\lambda_1$, \ldots, $\lambda_m$ такие, что
	$$\sum_{i=0}^{m}\lambda_i=1,\ \ \lambda_i\geq 0,\ \ i=\overline{1,m}, $$
	$$\lambda_0 \phi'(x_*)+\sum_{i=1}^{m}\lambda_i \psi_i'(x_*)=0,
	\lambda_i \psi_i(x_*)=0. $$
\end{theorem}

Для доказательства нам потребуется два вспомогательных утверждения. Прежде всего напомним, что если $Z\subset X$ некоторое множество, $v\in X$, то проекция $v$ на $Z$ есть элемент $p\in Z$ такой, что
$$\|v-p\|=\inf_{z\in Z}\|v-z\|. $$
\begin{lemma}\label{lm:convex}
	Пусть $Z$ выпуклое подмножество $\mathbb{R}^n$, $v\in\mathbb{R}^n$. Тогда $p$ является проекцией $x$ на $Z$ тогда и только тогда, когда для всех $z\in Z$
	$$(z-p)^T(v-p)\leq 0. $$
\end{lemma}
\begin{proof}
	Пусть $p$ -- проекция $v$ на $Z$. Заметим, что для всех $z\in Z$, $\alpha\in [0,1]$, $(1-\alpha)p+\alpha z \in Z$. Тогда
	\begin{equation*}
	\|(1-\alpha) p+\alpha z-v\|\geq \|p-v\|.
	\end{equation*} Отсюда,
	\begin{equation*}\|\alpha (z-p)-(v-p)\|=\alpha^2\|z-p\|^2+\|v-p\|^2-2\alpha (z-p)^T(v-p) \\ \geq \|p-v\|^2.\end{equation*} Сокращая члены, получаем, что
	$$\alpha^2\|z-p\|^2-2\alpha (z-p)^T(v-p)\geq 0. $$ Это неравенство выполнено для всех $\alpha\in [0,1]$. Отсюда, проанализировав изменение знака в левой части, заключаем, что $(z-p)^T(v-p)\leq 0$. 
	
	Обратно, для всех $z\in Z$ имеем, что
	\begin{multline*}
	\|z-x\|^2=\|(z-p)-(v-p)\|^2\\=\|z-p\|^2+\|v-p\|^2-2(z-p)^T(x-p)\geq \|v-p\|^2.
	\end{multline*}	 Отсюда следует, что $p$ -- проекция $v$ на $Z$.
\end{proof}

Для доказательства мы введем множество $K\subset X^*$ по правилу:
$$K\triangleq \left\{\lambda_0\phi'(x_*)+\sum_{i=1}^m\lambda_i\psi_i'(x_*):\lambda_i\geq 0,\ \ i=\overline{1,m},\ \ \sum_{i=0}^m\lambda_i=1\right\}. $$ Легко видеть, что множество $K$  выпукло. Теорема Куна-Такера говорит о том, что $0$ лежит в $K$.

\begin{proof}[Доказательство теоремы \ref{th:Kuhn_Tacker} для конечномерного случая]
 	Вначале мы рассмотрим случай, когда $X=\mathbb{R}^n$. Также мы для упрощения рассматриваем случай, когда для всех $i=\overline{1,m}$ $\psi_i'(x_*)=0$ (т.е. ограничения активны). В противном случае, если какое-то $\psi_i(x_*)<0$, положим $\lambda_i=0$. Заметим, что таким образом последнее условие всегда выполнено. Остается доказать два первых. 
 	
 	Предположим, что теорема Куна-Такера не верна т.е., что $0\notin K$. Тогда выберем $z=\phi'(x_*)$, $v=0$. Пусть $p$ проекция $v$ на $K$. Тогда по Лемме \ref{lm:convex}
 	$$\langle \phi'(x_*),p\rangle\geq \|p\|^2. $$ Аналогично, поскольку $\psi_i'(x_*)\in K$, используя Лемму \ref{lm:convex}, мы получаем, что
 	$$\langle \psi'_*(x_*), -p\rangle\leq -\|p\|^2<0. $$
 	
 	Отсюда,  воспользовавшись дифференцируемостью $\phi$ и $\psi_i$ в $x_*$, получаем, что
 	$$\phi(x_*-\alpha p)-\phi(x_*)=-\alpha \langle \phi'(x_*),p\rangle +O(\alpha^2), $$
 	$$\psi_i(x_*-\alpha p)-\psi_i(x_*)=-\alpha \langle \phi'(x_*),p\rangle +O(\alpha^2). $$ Таким образом, для достаточно малого $\alpha>0$ получаем, что $\psi_i(x_*-\alpha p)<0$, т.е. $\psi_i(x_*-\alpha p)$ допустима. В то же время,
 	$$\phi(x_*-\alpha p)-\phi(x_*)\leq -\frac{\alpha}{2}\|p\|^2<0. $$ Отсюда получаем противоречие: в допустимой точке $x_*-\alpha p$ достигается значение меньшее $\phi(x_*)$.
 	
 	
\end{proof}

Для того чтобы доказать теорему Куна-Такера в общем случае нам необходимо следующее вспомогательное утверждение. В его формулировке используется выпуклая оболочка системы векторов напомним, что она обозначается через $\mathrm{co}$
\begin{lemma}\label{lm:banach} Пусть $X$ -- бесконечномерное банахово пространство, $\xi_1,\ldots,\xi_m\in X^*$ таковы, что
	$$0\notin\mathrm{co}\{\xi_1,\ldots,\xi_m\}. $$ Тогда существуют $y_1,\ldots,y_m\in X$ такие, что
	$$0\notin\mathrm{co}\left\{\left(\begin{array}{c}\langle \xi_1,y_1\rangle \\ \cdots \\\langle \xi_1,y_m\rangle\end{array}\right),\ldots,\left(\begin{array}{c}\langle \xi_m,y_1\rangle \\ \cdots \\\langle \xi_m,y_m\rangle\end{array}\right)\right\}. $$
\end{lemma}
\begin{proof}
	Напомним, что традиционно $\mathrm{Ker}(p)$ обозначает множество элементов, которые переводятся в ноль т.е. $\mathrm{Ker}(p)\triangleq \{x\in X:\langle p,x\rangle=0\}$. Пусть 
	$$M\triangleq \bigcap_{i=1}^m\mathrm{Ker}(\xi_i). $$ Имеем, что для всех $x_1,\ldots,x_{m+1}$ существуют такие ненулевые $\mu_1$,\ldots, $\mu_{m+1}$, что
	$$\sum_{i=1}^{m+1}\mu_i x_i\in M. $$ В самом деле, система (относительно $\mu_i$)
	$$\sum_{i=1}^{m+1}\mu_i\langle \xi_j, x_i\rangle=0 \ \ j=1,\ldots, m+1$$ всегда имеет хотя бы одно нетривиальное решение. Это означает, что для всех $j$ $\sum_{i=1}^{m+1}\mu_i x_i\in \mathrm{Ker}(\xi_j)$. 
	
	Рассмотрим теперь набор точек $x_1,\ldots,x_m$ такой, что любая их невырожденная линейная комбинация не лежит в $M$ (если такой нет, то наше пространство конечномерно, что противоречит условиям леммы). Заметим, что выше мы доказали, что $M$ есть ортогональное дополнение к $\mathrm{span}\{x_1,\ldots,x_m\}. $ Если
	$$0\in  \mathrm{co}\left\{\left(\begin{array}{c}\langle \xi_1,x_1\rangle \\ \cdots \\\langle \xi_1,x_m\rangle\end{array}\right),\ldots,\left(\begin{array}{c}\langle \xi_m,x_1\rangle \\ \cdots \\\langle \xi_m,x_m\rangle\end{array}\right)\right\},  $$ то 
	для некоторого набора $\alpha_1$, \ldots, $\alpha_m$ такого, что 
	$$\alpha_i\geq 0,\ \ \sum_{j=1}^m \alpha_j=1 $$ выполнены равенства
	\begin{equation}\label{equality:alpha_xi_x}
	\left\langle\sum_{j=1}^m\alpha_j \xi_j,x_i\right\rangle=0,\ \ i=1,\ldots m. 
	\end{equation} Поскольку $M$ есть ортогональное дополнение к $\mathrm{span}\{x_1,\ldots, x_m\}$, любой вектор $x\in X$ может быть представлен в виде $x=\beta_1 x_1+\ldots+\beta_m x_m+z$, где $z\in M$, $\beta_i\in \mathbb{R}$. В силу определения $K$ и (\ref{equality:alpha_xi_x}) имеем, что
	$$\left\langle\sum_{j=1}^m\alpha_j \xi_j,x\right\rangle=\sum_{i=1}^m\left\langle \sum_{j=1}^m \alpha_j \xi_j,x_i\right\rangle+\left\langle \sum_{j=1}^m \alpha_j \xi_j,z\right\rangle=0. $$ Это означает, что $0=\sum_{j=1}^m\alpha_j \xi_j$. Это противоречит предположению лемму. Отсюда мы заключаем, что для любого набора $x_1,\ldots,x_m$, такого, что $\mathrm{span}\{x_1,\ldots,x_m\}\cap M=\{0\}$,
	$$0\notin  \mathrm{co}\left\{\left(\begin{array}{c}\langle \xi_1,x_1\rangle \\ \cdots \\\langle \xi_1,x_m\rangle\end{array}\right),\ldots,\left(\begin{array}{c}\langle \xi_m,x_1\rangle \\ \cdots \\\langle \xi_m,x_m\rangle\end{array}\right)\right\}.$$ Тем самым, лемма доказана (даже больше доказано).
\end{proof}

На основе этой леммы мы сможем доказать теорему Куна-Такера в общем случае.
\begin{proof}[Доказательство теоремы \ref{th:Kuhn_Tacker} в банаховом случае]
	Как и в конечномерном случае мы предполагаем, что $0\notin \mathrm{co}\{\phi'(x_*),\psi_1'(x_*),\cdot, \psi_m'(x_*)\}.$ Напомним, что $\phi'(x_*),\psi_1'(x_*),\ldots,\psi_m'(x_*)\in X^*$. По Лемме \ref{lm:banach} имеем, что существуют такие $y_0,y_1,\ldots,y_m$, что 
	$$0\notin \mathcal{K}\triangleq  \mathrm{co}\left\{\left(\begin{array}{c}
	\langle \phi'(x_*), y_0\rangle \\ \vdots \\ \langle \phi'(x_*), y_m\rangle	\end{array}\right),\left(\begin{array}{c}
	\langle \psi_1'(x_*), y_0\rangle \\ \vdots \\ \langle \psi_1'(x_*), y_m\rangle	\end{array}\right),\ldots, \left(\begin{array}{c}
	\langle \psi_m'(x_*), y_0\rangle \\ \vdots \\ \langle \psi_m'(x_*), y_m\rangle	\end{array}\right)\right\}.$$ Обозначим через $p$ проекцию $0$ на $\mathcal{K}$. Имеем, что $p$ -- это $(m+1)$-мерный вектор $p=(p_0,p_1,\ldots,p_m)^T$. Воспользуемся теперь Леммой \ref{lm:convex}. Она говорит, что для всех $z\in\mathcal{K}$ (заметьте, что $z$ это $(m+1)$-мерный вектор)
	$$z^T(-p)\leq -\|p\|^2. $$ Теперь выбирая в качестве $z$ вектор $(\langle \phi'(x_*), y_0,\rangle, \ldots,\langle \phi'(x_*),y_m\rangle)^T$ получаем, что
	\begin{equation}\label{ineq:banach_phi_prime}
	\sum_{j=0}^{m}\langle \phi'(x_*),- p_jy_j\rangle\leq -\|p\|^2.
	\end{equation} Аналогично, выбирая $z=(\langle \psi_i'(x_*), y_0,\rangle, \ldots,\langle \psi_i'(x_*),y_m\rangle)^T$, мы получаем, что
	\begin{equation}\label{ineq:banach_psi_i_prime}
	\sum_{j=0}^{m}\langle \psi_i'(x_*),- p_jy_j\rangle\leq -\|p\|^2.
	\end{equation} Обозначим $w\triangleq \sum_{j=0}^{m} p_j y_j$. Внося сумму в скалярные произведения в (\ref{ineq:banach_phi_prime}), (\ref{ineq:banach_psi_i_prime}), мы получаем, что
	\begin{equation}\label{ineq:deriv_phi_banach}
	\langle \phi'(x_*),-w\rangle\leq -\|p\|^2,
	\end{equation}
	\begin{equation}\label{ineq:deriv_psi_i_banach}
	\langle \psi_i'(x_*),-w\rangle\leq -\|p\|^2.
	\end{equation}
\end{proof}
Раскладывая функции $\alpha\mapsto \phi(x_*-\alpha w)$, $\alpha\mapsto\psi_i(x_*-\alpha w)$ в ряд Тейлора в форме Пеано до первого порядка включительно мы получаем, что для достаточно малого $\alpha$ 
$$\phi(x_*-\alpha w)<\phi(x_*), $$
$$\psi_i(x_*-\alpha w)<\psi_i(x_*)\leq 0. $$ Следовательно в точках $x_*-\alpha w$ значение функции $\phi$ строго меньше чем в $x_*$ и эти точки являются допустимыми. Это противоречит предположению теоремы о том, что $x_*$ -- локальный минимум в задаче условной оптимизации в форме неравенств. Тем самым, теорема доказана.

\chapter{Простейшая задача вариационного исчисления}
\section{Некоторые постановки}
Вариационное исчисление выросла из естественного желания понять структуру вещей, имея в виду что все многие явления могут быть сводятся к задаче минимизации. К примеру, принцип Ферма утверждает, что движение света идет по кратчайшей линии. Аналогично, во многих случаях предполагается, что тело стремится к минимуму потенциальной энергии. Прежде чем мы сформулируем строгие постановки мы рассмотрим некоторые наводящие примеры.

\subsection{Цепная линия}
Рассмотрим тяжелую нить (цепь) длины $l$, подвешенную на двух опорах. Вводем декартовы координаты и обозначим ось абсцисс через $t$, ось ординат через $x$. Также положим, что заданы координаты опор $(t_0,x_0)$ и $(t_1,x_1)$. Плотность цепи обозначим через $\rho$. Выделим малый фрагмент цепи длины $ds$ на расстоянии $s$ от начала, мы можем посчитать потенциальную энергию этого фрагмента
$$dU(s)=gx(s)\rho ds. $$ Тогда общая потенциальная энергия есть
$$U=\int dU(s)=\int_{0}^{l}g\rho x(s)ds. $$ Форма цепной линии определяется условием минимума потенциальной энергии.
Теперь вычислим $ds=\sqrt{1+\dot{x}^2(t)}dt.$ Длине $0$ соответствует координата $t_0$, а длине $l$ -- координата $t_1$. Тогда мы можем переписать формулу для потенциальной энергии
$$U=\int_{t_0}^{t_1}g\rho x(t)\sqrt{1+\dot{x}^2}dt. $$ Форма цепи определяется минимум этой энергии при условиях $x(t_0)=x_0$, $x(t_1)=x_1$.

\subsection{Задача о брахистохроне}


\bibliographystyle{abbrv}
\bibliography{ayv_mo}

\end{document}